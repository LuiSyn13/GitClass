\documentclass[a4paper,12pt]{article}
\usepackage[utf8]{inputenc}
\usepackage[spanish]{babel}
\usepackage{amsmath}
\usepackage{graphicx}
\usepackage{hyperref}

\title{Git Class}
\author{Martin Nicasio}
\date{\today}

\begin{document}

\maketitle

\begin{abstract}
Este es el resumen de la clase de GIT
\end{abstract}

\section{Introducción}
Aquí comienza la introducción de tu documento.

\section{Contenido Principal}
Este es el cuerpo principal del documento.

\subsection{Subsección}
Puedes dividir tu documento en subsecciones para mayor claridad.

\begin{equation}
E = mc^2
\end{equation}

\begin{equation}
    c^2 = a^2 + b^2
\end{equation}

\section{Conclusión}
En la conclusión puedes resumir los puntos principales de tu documento.

\newpage
\section{Primeros comandos}

\end{document}
